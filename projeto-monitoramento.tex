%%% LaTeX Template: Two column article
%%%
%%% Source: http://www.howtotex.com/
%%% Feel free to distribute this template, but please keep to referal to http://www.howtotex.com/ here.
%%% Date: February 2011

%%% Preamble
\documentclass[	DIV=calc,%
							paper=a4,%
							fontsize=12pt,%
							onecolumn]{scrartcl}	 					% KOMA-article class

\usepackage{lipsum}													% Package to create dummy text
\usepackage[english]{babel}										% English language/hyphenation
\usepackage[protrusion=true,expansion=true]{microtype}				% Better typography
\usepackage{amsmath,amsfonts,amsthm}					% Math packages
\usepackage[pdftex]{graphicx}									% Enable pdflatex
\usepackage[svgnames]{xcolor}									% Enabling colors by their 'svgnames'
\usepackage[hang, small,labelfont=bf,up,textfont=it,up]{caption}	% Custom captions under/above floats
\usepackage{epstopdf}												% Converts .eps to .pdf
\usepackage{subfig}													% Subfigures
\usepackage{booktabs}												% Nicer tables
\usepackage{fix-cm}													% Custom fontsizes
\usepackage[utf8]{inputenc}
\usepackage[top=2.5cm, bottom=2.5cm, left=2.5cm, right=2.5cm]{geometry}
\usepackage[ddmmyyyy]{datetime}
\addto\captionsenglish{%
	\renewcommand\tablename{Tabela}
	\renewcommand\figurename{Figura}
} 
 

 
%%% Custom sectioning (sectsty package)
\usepackage{sectsty}													% Custom sectioning (see below)
\allsectionsfont{%															% Change font of al section commands
	\usefont{OT1}{phv}{b}{n}%										% bch-b-n: CharterBT-Bold font
	}

\sectionfont{%																% Change font of \section command
	\usefont{OT1}{phv}{b}{n}%										% bch-b-n: CharterBT-Bold font
	}



%%% Headers and footers
\usepackage{fancyhdr}												% Needed to define custom headers/footers
	\pagestyle{fancy}														% Enabling the custom headers/footers
\usepackage{lastpage}	

% Header (empty)
\lhead{}
\chead{}
\rhead{}
% Footer (you may change this to your own needs)

%% ====================================
%% ====================================
%% mude o rodape  do projeto
%% ====================================
%% ====================================

\lfoot{\footnotesize \texttt{Cabeamento estruturado} \textbullet ~ Monitoramento Colaborativo}


\cfoot{}
\rfoot{\footnotesize página \thepage\ de \pageref{LastPage}}	% "Page 1 of 2"
\renewcommand{\headrulewidth}{0.0pt}
\renewcommand{\footrulewidth}{0.4pt}



%%% Creating an initial of the very first character of the content
\usepackage{lettrine}
\newcommand{\initial}[1]{%
     \lettrine[lines=3,lhang=0.3,nindent=0em]{
     				\color{DarkGoldenrod}
     				{\textsf{#1}}}{}}



%%% Title, author and date metadata
\usepackage{titling}															% For custom titles

\newcommand{\HorRule}{\color{DarkGoldenrod}%			% Creating a horizontal rule
									  	\rule{\linewidth}{1pt}%
										}

\pretitle{\vspace{-30pt} \begin{flushleft} \HorRule 
				\fontsize{50}{50} \usefont{OT1}{phv}{b}{n} \color{DarkRed} \selectfont 
				}

%% ====================================
%% ====================================
%% mude o titulo  do projeto
%% ====================================
%% ====================================

\title{Projeto de cabeamento estruturado - Monitoramento Colaborativo}					% Title of your article goes here

%% ====================================



\posttitle{\par\end{flushleft}\vskip 0.5em}

\preauthor{\begin{flushleft}
					\large \lineskip 0.5em \usefont{OT1}{phv}{b}{sl} \color{DarkRed}}
\author{Marcos Vinicius Alves Balsamo}  	% Author name goes here


\postauthor{\footnotesize \usefont{OT1}{phv}{m}{sl} \color{Black} 
					\\Universidade Tecnológica Federal do Paraná - Câmpus Cornélio Procópio 								% Institution of author
					\par\end{flushleft}\HorRule}

\date{}																				% No date




%%% Begin document
\begin{document}
\maketitle
\thispagestyle{fancy} 	
\thispagestyle{empty}		% Enabling the custom headers/footers for the first page 
% The first character should be within \initial{}




%% ====================================
%% ====================================
%% mude o resumo  do projeto
%% ====================================
%% ====================================
\initial{O}\textbf{s comerciantes e moradores da rua X indignados com a falta de segurança na cidade propuseram que fosse projetado um sistema de monitoramento colaborativo utilizando câmeras de alta resolução (câmera IP com resolução 1080 linhas).}

%% ====================================
\begin{figure}
	\centering
	\includegraphics{utfpr}
\end{figure}

\vspace{3cm}
\centerline{\textit{\textbf{\today}}}

\clearpage
    \renewcommand*\listfigurename{Lista de figuras}
\listoffigures

\renewcommand*\listtablename{Lista de tabelas}
\listoftables




\clearpage
\renewcommand{\contentsname}{Sumário}
\tableofcontents
\clearpage

%% ====================================
%% ====================================
%% Inicio do texto
%% ====================================
%% ====================================
\section{Introdução}
Um sistema de monitoramento utilizando câmeras IP exige planejamento no quesito infraestrutura de rede, no total são 10 clientes entre comerciantes e residentes, além disso pretendem acessar as câmeras via app de celular e a possibilidade de possuir para o uso de todos 2 links de internet de  100mb/100mb Fibra para acesso a internet além do acesso as câmeras que serão instaladas ao longo da rua, isso respeitando a privacidade das câmeras internas e de seus dispositivos locais que contabilizados somaram 10 roteadores sem fio, 50 celulares, 30 SmartTV's.

\subsection{Benefícios}
Os benefícios são o acesso de todos as câmeras instaladas na rua assim propiciando um monitoramento colaborativo, maior organização e gerencia no link de internet

\section{Estado atual}
Atualmente os clientes não estão interconectados, cada cliente tem contratação simples de internet com roteador sem fio fornecido pela prestadora.

Logo abaixo os passivos de rede, as principais reclamações e observações:
\begin{itemize}
	\item os passivos de rede atuais: roteadores sem fio e hub's;
	\item as principais reclamações dos usuários: os clientes utilizam o acesso wi-fi nas empresas porém os serviços e sistemas não estão isolados, constantemente estão recebendo ataques e desconexões em seus sistemas e serviços.
	\item Observações. Existe estruturas que não se enquadram nas normas e indicam suspeita de problemas a maioria dos cabos de rede em operação não foram crimpados conforme norma ANSI/TIA/EIA-568-A e B, assim ocasionando lentidão e o mau funcionamento da rede.
\end{itemize}


\section{Usuários e Aplicativos}
Cada vez mais a proporção é que a rede cresça seja em seu volume de dados quanto fisicamente, no próximos 8 meses a estimativa de 80 porcento de crescimento pelo alto volume de banda utilizada em jogos e streaming em redes sociais por exemplo o recurso "Ao vivo no Facebook".

\subsection{Usuários}
Comerciantes: Pizzaria, Escola de Dança, Loja de Baterias, Auto Peças, Farmácia. 
Moradores: Profissional Liberal, Funcionário Publico, Empresário, Aposentado.


\subsection{Aplicativos}
Aplicativos utilizados para segurança e monitoramento:
\begin{itemize}
	\item Denpa NVR: Sistema de gerenciamento de câmeras e gravação de imagens.
	\item Myeye - Aplicativo Android para acesso externo e interno via celular.
\end{itemize}

\section{Estrutura predial existente}
São 5 salas comerciais distribuídas em quadra de 100 metros com 5 residencias.

\section{Planta Lógica - Elementos estruturados}

\subsection{Estado atual}
Deve ter a planta atual, se for o caso

\subsection{Topologia}
A proposta futura, é que os estabelecimentos e as residencias estejam interligadas com link redundante, cada estabelecimento e residencias com sua VLAN própria, e as todas as câmeras na mesma VLAN. O gerenciamento de internet, firewall e o servidor dhcp estarão centralizados.

\begin{figure}
	\centering
	\includegraphics[]{rede}
	\caption{Topologia de rede}
	\label{rede}
\end{figure}

\subsection{Encaminhamento}
Manqueiras de ferro, calhas e eletrodutos.

\subsection{Memorial descritivo}
\begin{itemize}
\item 1x- Rack 19'' - Nilko
\item 1x- Patch Panel 24 Portas
\item 1x- Firewall - Palo Alto Networks - PA-3000 Series
\item 3x- Switch Gerenciável - SRW224G4-K9-BR
\item 1x- Modem fibra óptica
\end{itemize}


\subsection{Identificação dos cabos}

A identificação dos cabos será utilizando a tag ES + Número do Switch + Número do Ponto, por exemplo:
\begin{itemize}
\item ES 01 09
\end{itemize}
Ponto 09 que está ligado ao switch número 1.

\section{Implantação}
\begin{itemize}
\item 1 -Instalar cabeamento;
\item 2- Identificação dos pontos de rede
\item 3- Passagem de fibras ópticas, Fusão e Conectorização
\item 4 - Furação, fixação do Rack e canaletas.
\item 5- Conectorização dos pontos de acesso
\item 6- Certificação de pontos de acesso da rede de dados
\item 7- Montagem do Rack
\item 8 - Certificação de ponto;
\item 9 - Instalação de switch;
\item 10- Configuração de switch;
\item 11- Configuração de VLAN's.
\item 12- Instalação de NVR's;
\item 13- Instalação de Câmeras e regulagem de ângulo;
\item 14- Definir melhor ponto de instalação para roteadores sem fio através da função Site Survey.
\item 15- Instalação e configuração de roteadores sem fio.

\end{itemize}

\section{Plano de certificação}

As etapas são:
\begin{itemize}
\item 1- Utilização do fluke e do certificador de pontos para verificar se o cabo é adequado e não esta torcido ou possui emendas.  
\item 2- Toda rede será certificada, os locais e horários para execução da certificação na rede iniciam das 07:00 as 19:00.
\item 3- Será necessário apresentação dos relatórios de certificação de todos os pontos ao final de todos os testes.
\end{itemize}

\section{Plano de manutenção}

Revisões periódicas na rede, emissão de certificados para novos pontos.

\subsection{Plano de expansão}
Existe um plano de expansão? Quantos novos pontos poderão ser acrecidos na rede, antes de migração de equipamentos na camada 2? Se houver expansão, quais equipamentos deverão ser direcionados para as estremidades da rede? 


\section{Orçamento}
Crie uma relação de orçamentos baseado na seções anteriores.

\section{Referências bibliográficas}
Utilize o mendley, o jabref ou diretamente o bibtex para gerenciar suas referências biliográficas. As referências são criadas automaticamente de acordo com o uso no texto.

Exemplo: Redes de computadores, segundo \cite{t2013} é considerada..... Já \cite{kurose2010} apresenta uma versão...

Analisando os pressupostos de \cite{ref3} e \cite{ref4} concluimos que....


\renewcommand\refname{} %%Referências bibliográficas}  
\bibliographystyle{ieeetr}
\bibliography{referencias}  

%% ***********************************************************************
%% === remover daqui =====================================================
%% ***********************************************************************

\section{Elementos textuais - Alguns exemplos}

Esta seção apresenta exemplos de elementos textuais. \textbf{Remova-a da versão final do texto}.


\subsection{Colocar elementos em itens}

Texto antes da lista

\begin{itemize}
	\item First item in a list 
	\item Second item in a list 
	\item Third item in a list
\end{itemize}

\subsubsection{Uma sub seçao de terceiro nivel}

Exemplo de uma subseção

\subsection{Tabelas}

Utilize o site http://www.tablesgenerator.com/ para elaborar as tabelas de seu trabalho.
Para adicionar uma tabela utilize: a tag input, passando o arquivo da tabela como parametro

\input{tab2}

Dentro do arquivo você deve definir o label e pode utilizá-lo para referenciar. Exemplo:
Na tab \ref{tab2} temos a relação de ....


Você também pode modificar a tabela manualmente, incluindo, por exemplo h! dentro de sua definição. Veja no exemplo tab2.tex

\subsection{Figuras}

As figuras podem ser no formato PDF, JPG, PNG. Você pode referenciá-las da mesma maneira que tabelas. Exemplo: A figura \ref{fig1} apresenta.....

Não se preocupe o local em que a figura será renderizada em seu texto. Preocupe-se em criar referência para ela, ou seja, toda figura e tabela deve conter pelo menos uma referência no texto.

\begin{figure}
\centering
\includegraphics[width=\textwidth]{fig1}
\caption{Exemplo de figura com escala horizontal}
\label{fig1}
\end{figure}



Você pode rotacionar figuras também. Para isso utilize o parâmetro angle=-90. Repare que a escala da figura foi modificada pelo parametro height. Você também pode utilizar scale

\begin{figure}
	\centering
	\includegraphics[height=\textwidth,angle=-90]{fig3}
	\caption{Exemplo de figura rotacionada}
	\label{fig3}
\end{figure}


%% ***********************************************************************
%% === ate aqui    =====  ================================================
%% ***********************************************************************
\end{document}